\section{Literature Review}
\subsection{ Operation, Subsystems and Parameters}
\paragraph{}Condition monitoring is a type of maintenance inspection where an operational asset is monitored and the data obtained is analyzed to detect signs of degradation, diagnose the cause of faults, and predict for how long it can be safely or economically run. There are five general categories of Condition Monitoring techniques—vibration monitoring and analysis; visual inspection and nondestructive testing; performance monitoring and analysis; analysis of wear particles in lubricants and of contaminants in process fluids; and electrical plant testing.  Condition Monitoring needs good quality data such as that obtained by carefully run tests. However, much useful information can often be obtained from a plant's permanent instrumentation once repeatability is established.
\cite{thanapalan_model_2011}.

\paragraph{}Remaining Useful Life (RUL) is the time remaining for a component to perform its functional capabilities before failure.
The concept of Remaining Useful Life (RUL) is used to predict life-span of components (of a service system) with the purpose of minimising catastrophic failure events in both manufacturing and service sectors. Our proposal involves acquiring real data from a normal working pump at its different stages in life. This will be done using multiple sensors attached to the pump at different times under different working conditions. The data include temperature, vibration and pressure.
Over time, the installed sensors will generate more and more data which can be used to improve the initial models and make near-perfect failure predictions.


\paragraph{}Currently the industry is majorly relying on sensors for condition monitoring which has facilitated decision making under time constraints.
The time between the point where a potential failure occurs and the point where it deteriorates into a functional failure can be seen as an opportunity window during which decision making algorithms can recommend actions with the aim to eliminate the anticipated functional failure or mitigate its effect. The system can record and monitor vibration and temperature conditions of an industrial motor and transmit the data through a wireless network to a data logging center. The current prototype was developed using open source software and hardware and can successfully identify abnormal motor conditions from sensor input values that exceed predefined setpoints.


