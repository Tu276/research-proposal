\addcontentsline{toc}{section}{Abstract}

\section*{Abstract}
\label{sec:Abstract}
Maritime shipping is employed to move food, medicines, and far more at the heart of world trade. For growth and development, economical mechanisms of shipping are followed, particularly within the developing world. Machinery and equipment breakdown  that occur during marine transportation  cause delays, swings in supplies, production, and trigger infinite downstream effects on entire supply chains. In the shipping industry planned and reactive maintenance that is primarily  practiced requires halting of vessel operations for a number of days despite the fact that no fatigue or machinery and equipment breakdown is observed.
Having seen  the importance of motors that are widely utilized in ships auxiliary machinery, the project focuses on creating failure predictions, additionally determining  remaining useful  life (RUL) for motors aboard ships, albeit also quite fascinating is the role this research plays in the monitoring of equipment on autonomous ships. This is made possible with developments in information and technology. Massive amounts of data  is collected  and can facilitate condition based monitoring (CBM), conduct analysis on best performance and comprehensively diagnose ship engine room equipment.


\paragraph{\textbf{Keywords:}} efficiency, condition monitoring, remaining useful life, autonomous ships, performance.




